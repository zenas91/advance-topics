\documentclass[11pt,xcolor={dvipsnames},hyperref={pdftex,pdfpagemode=UseNone,hidelinks,pdfdisplaydoctitle=true},usepdftitle=false]{beamer}
\usepackage{presentation}
\usepackage{tasks}
% Enter title of presentation PDF:
\hypersetup{pdftitle={Presentation Title}}
% Enter link to PDF file with figures:
\newcommand{\pdf}{figures.pdf}

\begin{document}
% Enter presentation title:
\title{How to read and present a research paper}
% Enter presentation information:
\information%
% Enter link to research paper (optional; comment line if not needed):
%[https://github.com/zenas91/advance-topics]
% Enter presentation authors:
{Chukwuebuka Ezelu, Prof. Dr. Florian Wahl}%
% Enter presentation location and date (optional; comment line if not needed):
{20.10.2023}
\frame{\titlepage}

% Fill out content of presentation:
\begin{frame}
	\frametitle{Agenda}
	\begin{itemize}
		\item Reading a research paper
		\begin{itemize}
			\item Structure of scientific paper
			\item Three Pass Approach
		\end{itemize}
		\item Presenting a research paper
		\begin{itemize}
			\item Core Concepts and Scientific Contribution
			\item Result of the Scientific Study
			\item Author's Conclusions
			\item Personal Conclusions
		\end{itemize}
		\item Conclusion
	\end{itemize}
\end{frame}

\begin{frame}
	\heading{Reading a Resarch Paper}
\end{frame}

\begin{frame}
\frametitle{Structure of scientific paper}
While there can be variations in the structure of scientific papers, there is a fundamental framework that applies to various fields. These components typically include::
\begin{itemize}
\item Title
\item Abstract
\item Keywords
\item Introduction
\item Method
\item Result
\item Discussion
\item Reference
\end{itemize}
\end{frame}

\begin{frame}
	\frametitle{Abstract}
	This is the summary of the entire paper and contains four pieces of information, namely:
	\begin{enumerate}
		\item Rationale of the study:  This essentially defines the necessity for the study.
		\item Method: Explains the methodology adopted for the study. In other words, it explains how the study was conducted.
		\item Result: This is the observation of the study.
		\item Conclusion: The interpretation of the result or rather what the result means.
	\end{enumerate}
\end{frame}

\begin{frame}
	\frametitle{Introduction and Method}
	\begin{block}{Introduction}
		The introduction gives background information about the topic and sets out specific questions to be addressed by the authors. You can skim through the introduction if you are already familiar with the paper’s topic
	\end{block}
	\begin{block}{Method}
		The methods section gives technical details of how the experiments were carried out and serves as a “how-to” manual if you wanted to replicate the same experiments as the authors. This is another section you may want to only skim unless you wish to identify the methods used by the researchers or if you intend to replicate the research yourself.
	\end{block}
\end{frame}

\begin{frame}
	\frametitle{Results and Discussion}
	\begin{block}{Result}
		The results are the meat of the scientific article and contain all of the data from the experiments. You should spend time looking at all the graphs, pictures, and tables as these figures will contain most of the data.
	\end{block}
	\begin{block}{Discussion}
		Lastly, the discussion is the authors’ opportunity to give their opinions. Keep in mind that the discussions are the authors’ interpretations and not necessarily facts. It is still a good place for you to get ideas about what kind of research questions are still unanswered in the field and what types of questions you might want your own research project to tackle. 
	\end{block}
\end{frame}

\begin{frame}
	\frametitle{Three Pass Approach}
	\begin{block}{First Pass}
		The first pass is a quick scan to get a bird’s-eye view of the paper. You can also decide whether you need to do any more passes. This pass should take about five to ten minutes and consists of the following steps:
		\begin{itemize}
			\item Carefully read the title, abstract, and introduction
			\item Read the section and sub-section headings, but ignore
			everything else
			\item Read the conclusions
			\item Glance over the references, mentally ticking off the
			ones you’ve already read
		\end{itemize}
		At the end of the first pass, you should be able to answer
		the five Cs:
		\begin{tasks}[style=enumerate, item-format={\normalfont}, after-item-skip=4mm](4)
			\task Category 
			\task Context
			\task Correctness
			\task Contribution
			\task Clarity
		\end{tasks}
	\end{block}
	
\end{frame}

\begin{frame}
	\frametitle{Three Pass Approach cont.}
	
	\begin{block}{Second Pass}
		The second pass necessitates a more meticulous examination compared to the first pass. Read the paper attentively, paying closer attention to detail, while potentially disregarding intricate elements like proofs. It can be beneficial to take notes on the main points or annotate the margins as you progress through the text. In this pass, you should:
		\begin{itemize}
			\item Examine the figures, diagrams, and other illustrations in the paper meticulously. Give particular emphasis to graphs and charts.
			\item Identify pertinent references that you haven't read yet and mark them for further exploration. This can provide a deeper understanding of the paper's background and context.
		\end{itemize}
		
	\end{block}
\end{frame}

\begin{frame}
	\frametitle{Three Pass Approach cont.}
	After the first two passes, you should be able to summarize the paper's main points, contributions, and findings, providing evidence from the text to support your summary.
	\begin{block}{Third Pass}
		The third pass, a crucial step in paper understanding and review, entails a thorough virtual re-implementation of the work. By recreating the paper's content while assuming the author's perspectives, you can reveal innovations, hidden limitations, and underlying assumptions. This process demands meticulous attention to detail and the questioning of every statement and assumption, fostering a deep understanding of the paper's techniques. It's an opportunity to generate ideas for future research. The outcome should be an in-depth understanding of the paper's structure, strengths, weaknesses, implicit assumptions, citation gaps, and potential issues in methodologies.
	\end{block}
\end{frame}


\begin{frame}
	\heading{Presenting a Research Paper}
\end{frame}

\begin{frame}
	\frametitle{Presenting a Research Paper}
	When presenting a scientific paper, it's essential to cover four key areas, as explained below:
	
	\begin{block}{1. Core Concepts and Scientific Contribution}
		Clearly articulating the paper's core contributions is essential, as it demonstrates a profound grasp of the topic and a clear awareness of its significance in the scientific domain.
	\end{block}
	
	\begin{block}{2. Result of the Scientific Paper}
		While stating the contributions could show an understanding of the topic, being able to judge the result of an experiment is an indispensable tool. It provides the foundation for the experiment conclusions that will follow.
	\end{block}
\end{frame}

\begin{frame}
	\frametitle{Presenting a Research Paper cont.}
	\begin{block}{3. Author's Conclusions}
		The discussion and conclusion of most scientific papers contain the author's deductions based on their experimental results. This is a crucial component of presenting any scientific work as it declares whether the hypothesis was supported or refuted.
	\end{block}
	\begin{block}{4. Personal Conclusions}
		This is particularly important when reviewing the work of others. Personal views can encompass various aspects, from evaluating the paper's structure to assessing the writing style. It also necessitates a solid understanding of the field to discern the validity of the paper's hypothesis and claims.
	\end{block}
\end{frame}

\begin{frame}
	\heading{Questions?}
\end{frame}

\lastslide



\end{document}